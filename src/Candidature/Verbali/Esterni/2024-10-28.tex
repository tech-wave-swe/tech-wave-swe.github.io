\documentclass{TWReport}

\title{Verbale 28 ottobre 2024 - Ergon}

\author{Giulia Marcon}
\participant[present]{Carraro Agnese}
\participant[present]{Dal Bianco Riccardo}
\participant[present]{Marcon Giulia}
\participant[present]{Monetti Luca}
\participant[present]{Piola Andrea}
\participant[present]{Pistori Gaia}
\participant[absent]{Vasquez Manuel Felipe}

\editor{Giulia Marcon}

\reviewer{Luca Monetti}
\classification{Esterno}
\version{1.0}

\begin{document}

\frontmatter

\showPartecipants

 \begin{table}[h]
  \centering
  \renewcommand{\arraystretch}{1.5}
    \rowcolors{1}{}{twlightblue}%
        \begin{tabularx}{\textwidth}{|>{\centering\arraybackslash}X|>{\centering\arraybackslash}X|>{\centering\arraybackslash}X|>{\centering\arraybackslash}X|>{\centering\arraybackslash}X|>{\centering\arraybackslash}X|}
        \hline
        \textbf{Data Modifica} & \textbf{Versione} & \textbf{Annotazione} & \textbf{Autore Modifica} & \textbf{Data Approvazione} & \textbf{Autore Approvazione} \\
        \hline
        {28/10/2024} & {1.0} & {prima stesura del documento} & {Giulia Marcon} & {29/10/2024} & {Luca Monetti} \\
        \hline
    \end{tabularx}
\end{table}

\section*{Ordine del giorno}
Meeting con l'azienda per chiarimenti su:
\begin{itemize}
    \item Fase di preprocessing e gestione dei dati errati.
    \item Collegamento tra i dati processati nel database e i file originali.
    \item Struttura dell'interfaccia web per la configurazione degli utenti.
    \item Template delle domande e modalità di suggerimento per l’utente.
    \item Gestione dei feedback degli utenti.
    \item Metodi di comunicazione tra il team e l'azienda durante lo svolgimento del progetto.
\end{itemize}

\section*{Resoconto}
\begin{itemize}
    \item È stato chiarito che la fase di preprocessing include il rilevamento di formati errati, lingue non riconosciute o documenti vuoti. In caso di dati errati, verranno scartati automaticamente o, dove possibile, verranno corretti manualmente.
    \item Eventuali modifiche sui file originali comporterebbero l’aggiornamento dei dati nel database, con possibilità di riaddestramento dell'assistente (opzionale).
    \item Il login dell'interfaccia web deve consentire agli amministratori (admin) di approvare nuovi utenti e permettere a utenti generici di accedere all’assistente virtuale. Saranno previste funzionalità per la creazione e gestione di utenti, l'addestramento del modello AI e la gestione dei template nel backend.
    \item Durante l’uso, l'assistente virtuale confronterà ogni domanda posta dall'utente con dei template: se trova una corrispondenza, genererà una risposta basata su quel template; in caso contrario, produrrà autonomamente una risposta per l’utente.
    \item L’assistente virtuale sarà strutturato su un unico flusso di conversazione, in cui le interazioni non verranno salvate a meno che l'utente non fornisca un feedback. Il feedback sarà possibile tramite un sistema di valutazione semplice (pollice su/giù o voto numerico) e potrà includere un commento testuale. Gli admin potranno monitorare questi feedback per valutare l’efficacia del sistema.
    \item Le risposte saranno testuali, con possibilità di integrazione di immagini (opzionale).
    \item L'azienda fornirà dati specifici per costruire il contenuto dell'assistente e un pacchetto di domande che l’utente potrà porre, centrato sul settore beverage/food.
    \item Il progetto prevede lo sviluppo dell’assistente virtuale tramite una web app dedicata al backoffice e un’applicazione per la chat, che sarà sviluppata come app mobile o come web app.
    \item Si è ipotizzato un incontro settimanale, in presenza o su Zoom, con la possibilità di variazione tramite comunicazione via email. Il team Ergon fornirà supporto attraverso corsi offline sulle tecnologie richieste e affiancherà il team durante il progetto.
\end{itemize}

\signature{L'Azienda Proponente}

\end{document}
